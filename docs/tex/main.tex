\documentclass{bmstu}



\usepackage{listings}
\usepackage{xcolor}

% Настройка стиля листингов
% \lstset{
%   basicstyle=\ttfamily\footnotesize,
%   language=SQL, % Язык программирования
%   backgroundcolor=\color{white}, % Цвет фона
%   commentstyle=\color{green}, % Стиль комментариев
%   keywordstyle=\color{blue}, % Стиль ключевых слов
%   numberstyle=\tiny\color{black}, % Стиль номеров строк
%   numbers=left, % Размещение номеров строк
%   frame=single, % Обрамление листинга
%   breaklines=true, % Разрыв длинных строк
%   tabsize=4, % Размер табуляции
% }


\lstset{
    basicstyle=\ttfamily\footnotesize, % Основной стиль шрифта и размер
    commentstyle=\color{green}, % Стиль комментариев
    keywordstyle=\color{blue}, % Стиль ключевых слов
    numbers=left, % Показывать номера строк слева
    numbersep=3mm, % Отступ номеров строк
    numberstyle=\scriptsize, % Размер номеров строк
    breaklines=true, % Разрешить автоматический перенос строк
    frame=single, % Отображать рамку вокруг листинга
    rulecolor=\color{black}, % Цвет рамки
    showstringspaces=false, % Не отображать пробелы в строках
}

\bibliography{lib.bib}

\begin{document}

\makecourseworktitle
{Информатика и системы управления (ИУ)}
{Программное обеспечение ЭВМ и информационные технологии (ИУ7)}
{Разработка статического сервера}
{И. О. Артемьев/ИУ7-75Б}
{Н. О. Яковидис}
{}
{}
{}

\chapter*{РЕФЕРАТ}

В рамках данной курсовой работы был реализован статик-сервер.

Ключевые слова: статик-сервер, prefork, select, http.

Рассчетно-пояснительная записка к курсовой работе содержит \begin{NoHyper}\pageref{LastPage}\end{NoHyper} страниц, \totfig~иллюстраций, 4 источника.

\clearpage

\maketableofcontents

% \chapter{Аналитическая часть}

% Шифровальная машина <<Энигма>> состоит из трех основных частей:
% \begin{itemize}
%     \item роторы --- диски обладающие 26 гранями, где каждая грань представляла собой нумерацию английского алфавита;
%     \item рефлектор --- статический механизм, позволящий машине также расшифровать текст;
%     \item коммутатор --- набор парных шифров.
% \end{itemize}

% \section{Алгоритм работы машины}

% На вход <<Энигме>> подается строка, которая разбивается на символы. 
% Далее символ проходит через коммутационную панель, который меняет символ в соотвествии с настройкой.
% После прохождения панели, символ проходит через три диска и попадает на рефлектор.
% После работы рефлектора, символ отправляется обратно на диске и оканчательно шифруется через коммутатор.
% Затем один ротор совершает оборот, если ротор обернулся 26 раз, то поворачивается следующий.

% \section{Анализ залупы}

\chapter*{ВВЕДЕНИЕ}
\addcontentsline{toc}{chapter}{ВВЕДЕНИЕ}

Цель проекта состоит в создании статического веб-сервера, способного обрабатывать GET и HEAD запросы. 
Статический веб-сервер –- это программное решение, предназначенное для обслуживания и 
передачи статического контента по сети Интернет. Он обеспечивает доступ к различным ресурсам, 
таким как HTML-страницы, изображения, таблицы стилей и другие, которые не генерируются 
динамически на сервере.

В свете развития веб-технологий статические веб-серверы становятся важными, обеспечивая надежную и 
быструю обработку статических ресурсов. Они востребованы благодаря своей простоте и эффективности в 
предоставлении пользователям быстрого доступа к статическим компонентам веб-приложений.

Разработка статических веб-серверов включает в себя задачи по обеспечению безопасности, таких как 
предотвращение выхода за пределы определенной директории. Потенциальные угрозы могут привести к 
утечке конфиденциальных данных и другим негативным последствиям для сервера. В данной работе будет 
уделено внимание методам обеспечения безопасности статического веб-сервера, включая контроль 
доступа и предотвращение уязвимостей, связанных с управлением файловой системой сервера.

Для достижения поставленной цели, предполагается выполнение следующих задач:

\begin{itemize}
	\item провести формализацию задачи и определить небоходимый функционал;
	\item исследовать предметную область веб-серверов;
	\item спроектировать приложение;
	\item реализовать приложение;
	\item протестировать приложение на предмет корректности.
\end{itemize}

\chapter{Аналитический раздел}

\section{Требования к серверу}

\begin{itemize}
	\item поддержка запросов GET и HEAD (поддержка статусов 200, 403, 404);
	\item ответ на неподдерживаемые запросы статусом 405;
	\item выставление content type в зависимости от типа файла (поддержка .html, .css, .js, .png, .jpg, .jpeg, .swf, .gif);
	\item корректная передача файлов размером в 100мб;
	\item сервер по умолчанию должен возвращать html-страницу на выбранную тему с css-стилем;
	\item учесть минимальные требования к безопасности статик-серверов (предусмотреть ошибку в случае если адрес будет выходить за root директорию сервера);
	\item реализовать логгер;
	\item использовать язык Си, сторонние библиотеки запрещены;
	\item реализовать архитектуру prefork + select();
	\item статик сервер должен работать стабильно.
\end{itemize}

\section{Протокол HTTP}

Изучение современных веб-технологий требует глубокого понимания протокола HTTP (Hypertext Transfer Protocol). Этот стандарт является ключевым для взаимодействия между веб-клиентами и серверами, обеспечивая передачу данных в формате гипертекста. HTTP сыграл значительную роль в развитии интернета, служа основой для передачи ресурсов, таких как HTML-страницы, изображения и стили.

Протокол HTTP основан на принципе запроса-ответа, где клиент отправляет запрос на сервер для получения определенного ресурса, и сервер отвечает соответствующим ответом. Этот обмен сообщениями основан на простых принципах и стандартах, обеспечивая эффективное взаимодействие между веб-клиентами и серверами.

Существует несколько версий протокола HTTP, каждая из которых вносит улучшения и новшества. 
В данном контексте выбран HTTP 1.1 из-за его возможности многократного использования соединений 
(connection reuse). Эта характеристика позволяет снизить задержки при обмене данными, поскольку 
соединение может быть переиспользовано для последующих запросов, экономя трафик и улучшая производительность \cite{tanenbaum2024}. 

\section{Сокеты UNIX}

Сетевые сокеты представляют собой универсальный механизм для обмена данными между процессами на одной машине или по сети. В контексте создания веб-сервера, использование сетевых сокетов становится ключевым элементом, обеспечивая взаимодействие между сервером и клиентами.

Дуплексная связь, обеспечиваемая сетевыми сокетами, позволяет полнодуплексный обмен данными между сервером и клиентом. Это важный аспект в контексте веб-сервера, поскольку позволяет одновременно обрабатывать входящие и исходящие запросы, что способствует повышению производительности и отзывчивости сервера.

Файловые сокеты, также известные как доменные сокеты UNIX, играют важную роль в локальной сетевой коммуникации. Они предоставляют способ для процессов обмениваться данными напрямую, обеспечивая эффективную и безопасную передачу информации между различными частями приложения.

Выбор сетевых сокетов для реализации данного веб-сервера был обоснован их универсальностью и способностью обработки запросов как через сеть, так и локально. Сетевые сокеты обеспечивают надежное взаимодействие между сервером и клиентами, что является важным аспектом для корректной работы веб-сервера \cite{olifer2016}.

\section{Вывод}

На основании анализа были определены требования к статическому веб-серверу. Решение использовать протокол HTTP 1.1 было обосновано его возможностью эффективного переиспользования соединений. Также было обосновано предпочтение сетевым сокетам UNIX перед файловыми.

\chapter{Конструкторский раздел}

\section{HTTP-запросы}

Запрос в протоколе HTTP представляет собой структурированный набор строк. 
Начиная с первой строки, содержащей информацию о методе, URL и версии HTTP, 
последующие строки содержат заголовки запроса, которые определяют дополнительные 
параметры запроса. В некоторых случаях запрос может также содержать тело, которое 
используется для передачи дополнительных данных для более сложных веб-запросов.

\begin{lstlisting}[caption={Пример HTTP-запроса через Postman}, label=lst:http]
GET /img/prison.jpeg HTTP/1.1
User-Agent: PostmanRuntime/7.35.0
Accept: */*
Postman-Token: 7af874dc-5a4a-4678-a088-65bc9ca2b09e
Host: localhost:5600
Accept-Encoding: gzip, deflate, br
Connection: keep-alive
\end{lstlisting}


\section{HTTP-ответы}

Ответ в протоколе HTTP содержит информацию о версии HTTP, числовом представлении статуса, текстовом описании статуса, заголовках и теле ответа. Ответ может включать как текстовую, так и бинарную информацию. В случае передачи изображений или других бинарных файлов, которые требуется разбить на несколько буферов для эффективной передачи, особенно при больших размерах, данная операция выполняется с целью оптимизации передачи данных клиенту.

\section{Асинхронность}
Предлагается предоставить опцию использования серверной реализации, основанной на механизме prefork 
в сочетании с select для обработки входящих подключений. Этот подход позволяет создать множество 
процессов-обработчиков заранее (prefork) и использовать select для мониторинга активности сокетов и 
эффективной обработки ввода-вывода.

\section{Безопасность}

Большое внимание уделяется обеспечению безопасности статического веб-сервера. Основной угрозой является 
возможность несанкционированного доступа к файлам, расположенным за пределами директории, доступной для сервера.
Для защиты от доступа за пределами статической директории предлагается использовать функцию realpath, которая помогает получить абсолютный путь к файлу или директории. Это позволяет проверить, что запрашиваемый путь принадлежит к пределам статической директории сервера и предотвратить доступ к файлам за ее пределами.

\section{Вывод}
В данном разделе были сформулированы требования к безопасности, асинхронности, формату запросов и ответов.

\chapter{Технологический раздел}

Этот раздел охватит технические детали реализации приложения и продемонстрирует работу программы.

\section{Средства реализации}

Для написания программы был выбран язык программирования Си, который был рекомендован преподавателем. Си 
является низкоуровневым языком, обеспечивающим полный контроль над сокетами. Большую часть функционала 
пришлось реализовать вручную.
Для разработки программы использовалась среда Visual Studio Code \cite{vscode}.

\section{Листинги кода}

На листинге \ref{lst:server} представлена реализация алгоритма prefork + select.

\begin{lstlisting}[caption={Механизм prefork + select}, label=lst:server]
void start_server(void) {
    int server_socket;
    struct sockaddr_in server_addr;

    if ((server_socket = socket(AF_INET, SOCK_STREAM, 0)) == -1) {
        perror("Socket creation failed");
        exit(EXIT_FAILURE);
    }

    server_addr.sin_family = AF_INET;
    server_addr.sin_addr.s_addr = INADDR_ANY;
    server_addr.sin_port = htons(PORT);

    if (bind(server_socket, (struct sockaddr*)&server_addr, sizeof(server_addr)) == -1) {
        perror("Bind failed");
        exit(EXIT_FAILURE);
    }

    if (listen(server_socket, SOMAXCONN) == -1) {
        perror("Listen failed");
        exit(EXIT_FAILURE);
    }

    log_message(LOG_INFO, "Server listening on port %d...\n", PORT);

    for (int i = 0; i < NUM_CHILDREN; ++i) {
        pid_t pid = fork();

        if (pid == 0) {
            int client_sockets[MAX_CLIENTS] = { 0 };
            int max_fd, activity, addr_len, client_socket;
            struct sockaddr_in client_addr;
            fd_set readfds;

            while (1) {
                FD_ZERO(&readfds);
                FD_SET(server_socket, &readfds);

                max_fd = server_socket;

                for (int j = 0; j < MAX_CLIENTS; ++j) {
                    if (client_sockets[j] > 0) {
                        FD_SET(client_sockets[j], &readfds);
                        if (client_sockets[j] > max_fd) {
                            max_fd = client_sockets[j];
                        }
                    }
                }

                activity = select(max_fd + 1, &readfds, NULL, NULL, NULL); 
                if ((activity < 0) && (errno != EINTR)) {
                    perror("Select error");
                    exit(EXIT_FAILURE);
                }

                if (FD_ISSET(server_socket, &readfds)) {
                    addr_len = sizeof(client_addr);
                    if ((client_socket = accept(server_socket, (struct sockaddr*)&client_addr, (socklen_t*)&addr_len)) == -1) {
                        perror("Accept failed");
                        continue;
                    }

                    printf("New connection, socket fd is %d, ip is : %s, port : %d\n", client_socket,
                        inet_ntoa(client_addr.sin_addr), ntohs(client_addr.sin_port));

                    for (int j = 0; j < MAX_CLIENTS; ++j) {
                        if (client_sockets[j] == 0) {
                            client_sockets[j] = client_socket;
                            break;
                        }
                    }
                }

                for (int j = 0; j < MAX_CLIENTS; ++j) {
                    if (client_sockets[j] > 0 && FD_ISSET(client_sockets[j], &readfds)) {
                        handle_client(client_sockets[j]);
                        close(client_sockets[j]);
                        client_sockets[j] = 0;
                    }
                }
            }
        }
        else if (pid > 0) {
        }
        else {
            perror("Fork failed");
            exit(EXIT_FAILURE);
        }
    }

    for (int i = 0; i < NUM_CHILDREN; ++i) {
        wait(NULL);
    }
}
\end{lstlisting}

На листинге \ref{lst:request} представлена функция обработки HTTP-запроса, парсинг его и проверка на валидность.

\begin{lstlisting}[caption={Обработка и парсинг HTTP-запроса}, label=lst:request]
void handle_client(int client_socket) {
    char buffer[1024];
    char method[10], path[PATH_MAX], protocol[10];

    if (recv(client_socket, buffer, sizeof(buffer), 0) <= 0) {
        perror("Receive failed");
        return;
    }

    sscanf(buffer, "%s %s %s", method, path, protocol);

    if (strcmp(method, "GET") != 0 && strcmp(method, "HEAD") != 0) {
        send_error(client_socket, METHOD_NOT_ALLOWED);
        return;
    }

    char full_path[PATH_MAX];
    sprintf(full_path, "%s%s", STATIC_ROOT, path);

    char resolved_path[PATH_MAX];
    if (realpath(full_path, resolved_path) == NULL) {
        send_error(client_socket, NOT_FOUND);
        return;
    }
    if (strncmp(resolved_path, STATIC_ROOT, strlen(STATIC_ROOT)) != 0) {
        send_error(client_socket, FORBIDDEN);
        return;
    }

    if (strcmp(path, "/") == 0) {
        strcat(full_path, "index.html");
    }

    struct stat path_stat;
    if (stat(full_path, &path_stat) == 0 && S_ISDIR(path_stat.st_mode)) {
        if (path[strlen(path) - 1] != '/') { strcat(path, "/"); }
        send_directory_listing(client_socket, path, full_path);
        return;
    }

    send_file(client_socket, full_path, method);
}
\end{lstlisting}

На листинге \ref{lst:response} представлены функции обработки HTTP-ответов. Составление хедера и тела ответа в зависимости от ситуации.

\begin{lstlisting}[caption={Обработка HTTP-ответов}, label=lst:response]
void send_directory_listing(int client_socket, const char* current_path, const char* directory_path) {
    DIR* directory = opendir(directory_path);
    if (directory == NULL) {
        send_error(client_socket, NOT_FOUND);
        return;
    }

    char headers[1024];
    sprintf(headers, "HTTP/1.1 200 OK\r\nContent-Type: text/html\r\n\r\n");
    send(client_socket, headers, strlen(headers), 0);

    char buffer[1024];
    sprintf(buffer, "<html><head><title>Directory Listing</title></head><body><h1>Directory Listing</h1><ul>");

    struct dirent* dir_entry;
    while ((dir_entry = readdir(directory)) != NULL) {
        if (strcmp(dir_entry->d_name, ".") == 0) { continue; }

        char full_entry_path[255];
        sprintf(full_entry_path, "%s/%s", directory_path, dir_entry->d_name);

        struct stat path_stat;
        if (stat(full_entry_path, &path_stat) == 0 && S_ISDIR(path_stat.st_mode)) {
            sprintf(buffer + strlen(buffer), "<li><a href=\"%s%s/\">%s/</a></li>", current_path, dir_entry->d_name, dir_entry->d_name);
        }
        else {
            sprintf(buffer + strlen(buffer), "<li><a href=\"%s%s\">%s</a></li>", current_path, dir_entry->d_name, dir_entry->d_name);
        }
    }

    strcat(buffer, "</ul></body></html>");

    send(client_socket, buffer, strlen(buffer), 0);

    closedir(directory);
}

void send_file(int client_socket, const char* file_path, const char* method) {
    FILE* file = fopen(file_path, "rb");
    if (file == NULL) {
        send_error(client_socket, NOT_FOUND);
        return;
    }

    fseek(file, 0, SEEK_END);
    long file_size = ftell(file);
    fseek(file, 0, SEEK_SET);

    char headers[1024];
    sprintf(headers, "HTTP/1.1 200 OK\r\nContent-Length: %ld\r\n", file_size);

    const char* content_type = "text/plain";
    const char* file_ext = strrchr(file_path, '.');
    if (file_ext != NULL) {
        if (strcmp(file_ext, ".html") == 0) {
            content_type = "text/html";
        }
        else if (strcmp(file_ext, ".css") == 0) {
            content_type = "text/css";
        }
        else if (strcmp(file_ext, ".js") == 0) {
            content_type = "application/javascript";
        }
        else if (strcmp(file_ext, ".png") == 0) {
            content_type = "image/png";
        }
        else if (strcmp(file_ext, ".jpg") == 0 || strcmp(file_ext, ".jpeg") == 0) {
            content_type = "image/jpeg";
        }
        else if (strcmp(file_ext, ".swf") == 0) {
            content_type = "application/x-shockwave-flash";
        }
        else if (strcmp(file_ext, ".gif") == 0) {
            content_type = "image/gif";
        }
    }

    sprintf(headers + strlen(headers), "Content-Type: %s\r\n\r\n", content_type);


    send(client_socket, headers, strlen(headers), 0);

    if (strcmp(method, "HEAD") == 0) {
        fclose(file);
        return;
    }

    char buffer[1024];
    size_t bytes_read;
    while ((bytes_read = fread(buffer, 1, sizeof(buffer), file)) > 0) {
        send(client_socket, buffer, bytes_read, 0);
    }

    fclose(file);
}

void send_error(int client_socket, int status_code) {
    char response[1024];
    sprintf(response, "HTTP/1.1 %d\r\nContent-Length: 0\r\nContent-Type: text/html\r\n\r\n", status_code);

    log_message(LOG_INFO, "ERROR %s", response);
    log_message(LOG_DEBUG, "Debug message: %d", 42);
    // log_message(LOG_WARNING, "Something unexpected happened!");
    // log_message(LOG_ERROR, "Error occurred: %s", "File not found");

    send(client_socket, response, strlen(response), 0);
    // log_message("Sent forbidden error response to client");
}
\end{lstlisting}

На листинге \ref{lst:logger} представлена кастомная функция для логирования сервера и вывода всех логов в специально отведенный файл.

\begin{lstlisting}[caption={Логгер}, label=lst:logger]
void log_message(LogLevel level, const char* format, ...) {
    time_t rawtime;
    struct tm* timeinfo;
    time(&rawtime);
    timeinfo = localtime(&rawtime);
    LogLevel current_log_level = LOG_LEVEL;

    if (level > current_log_level) { return; }

    FILE* log_file = fopen(LOG_FILE, "a");
    if (log_file == NULL) {
        perror("Error opening log file");
        return;
    }

    char time_str[50];
    strftime(time_str, sizeof(time_str), "[%Y-%m-%d %H:%M:%S]", timeinfo);

    va_list args;
    va_start(args, format);

    fprintf(log_file, "%s ", time_str);

    switch (level) {
    case LOG_DEBUG:
        fprintf(log_file, "[DEBUG] ");
        break;
    case LOG_INFO:
        fprintf(log_file, "[INFO] ");
        break;
    case LOG_WARNING:
        fprintf(log_file, "[WARNING] ");
        break;
    case LOG_ERROR:
        fprintf(log_file, "[ERROR] ");
        break;
    }

    vfprintf(log_file, format, args);

    va_end(args);
    fclose(log_file);
}
\end{lstlisting}

\section{Демонстрация работы программы}

На рисунке \ref{img:index} показана работа home директории сервера. Сервер посвящен тематике капибар и мемам с ними.

\imgs{index}{h!}{0.24}{index.html}
\clearpage

На рисунке \ref{img:directory} показано корректное отображение списка файлов в запрашиваемой директории.

\imgs{directory}{h!}{0.24}{Директория img}

На рисунке \ref{img:meme} показана отдача фото с сервера.

\imgs{meme}{h!}{0.24}{Мем с капибарой}

\section{Вывод}

Был реализован статик-сервер на языке Си, а также проведены проверки его функционала.

\chapter{Исследовательский раздел}

\section{Описание эксперимента}
В ходе исследования было выполнено нагрузочное тестирование разработанного статик-сервера и nginx с использованием программы 
ali \cite{ali}. Тестирование было направлено на проверку отправки запросов к index.html.

\section{Результаты эксперимента}
На рисунках \ref{img:100 my} -- \ref{img:1000 nginx} рассматривались сценарии 100, 500 и 1000 запросов пользователей каждую секунду в течение 5 секунд.

\imgs{100 my}{h!}{0.47}{100 запросов в секунду к index.html на моем сервере}
\imgs{100 nginx}{h!}{0.47}{100 запросов в секунду к index.html на nginx сервере}
\imgs{500 my}{h!}{0.47}{500 запросов в секунду к index.html на моем сервере}
\imgs{500 nginx}{h!}{0.47}{500 запросов в секунду к index.html на nginx сервере}
\imgs{1000 my}{h!}{0.47}{1000 запросов в секунду к index.html на моем сервере}
\imgs{1000 nginx}{h!}{0.47}{1000 запросов в секунду к index.html на nginx сервере}
\clearpage

\section{Вывод}
Проведено тестирование разработанного статик-сервера и nginx, в ходе которого выявлено, что nginx работает эффективнее.


\chapter*{ЗАКЛЮЧЕНИЕ}
\addcontentsline{toc}{chapter}{ЗАКЛЮЧЕНИЕ}

В процессе выполнения курсовой работы был выполнен анализ предметной области, 
выделены требования к разрабатываемому приложению. Было разработано приложение, 
которое прошло успешное тестирование на корректность работы программы. Также 
были проведены тесты на нагрузку для оценки поведения приложения при высокой нагрузке.

При написании данной работы:

\begin{itemize}
	\item проведена формализация задачи и определен необходимый функционал;
	\item исследована предметная область веб-серверов;
	\item спроектировано приложение;
	\item реализовано приложение;
	\item приложение протестировано на предмет корректности.
\end{itemize}

Таким образом, поставленные задачи были выполнены, цель курсовой работы работы была достигнута.

% \begin{thebibliography}{9} % Обычно используются однозначные числа, чтобы уместить до 9 источников
%     \bibitem{label1} Автор1. Название1. Издание1. Год1.
%     \bibitem{label2} Автор2. Название2. Издание2. Год2.
%     % Добавьте больше элементов в формате \bibitem{label} ...
    
%     \end{thebibliography}

\clearpage
\makebibliography


% 3 ms дефолт
% 1 ms индексы
% 0.09 ms материализованная вьюха
% 3.5 ms предвыборка

% \include{01-definitions}
% \include{02-abbreviations}

% \include{03-introduction}
% \include{04-analysis}
% \include{05-design}
% \include{06-implementation}
% \include{07-research}
% \include{08-conclusion}

% \include{09-appendix}

\end{document}